\section{Technische Grundlagen}
\subsection{DCF77}
DCF77 ist ein Zeitzeichensender in Mainhausen bei Frankfurt, der seit dem ersten Januar 1959 die Uhrzeit auf der Langwellenfrequenz von 77,5 kHz sendet. Bei Sendeanlagen, die über Ländergrenzen hinaus senden, muss das Rufzeichen in der internationalen Frequenzliste eingetragen sein und das Kennzeichen des jeweiligen Landes enthalten. Deshalb wurde DCF77 gewählt, wobei das D für Deutschland steht. Der Buchstabe C war früher ein Kennzeichen für Langwelle und das F steht für Frankfurt. Da die Sendeanlage in Mainhausen mehrere Sender hat, wurde noch die Zahl 77 als Andeutung auf die Trägerfrequenz (77,5 kHz) gewählt\footnote{\cite{dcf77}, Seite 349f.}.

Das Zeitsignal wird jede Minute wiederholt und kodiert Zeit- sowie Datumsinformationen. Dabei wird jede Sekunde ein Bit übertragen. Dies geschieht durch Amplitudenmodulation. Jede Sekunde wird die Amplitude der Trägerwelle für 100 ms (logisch 0) oder 200 ms (logisch 1) auf etwa 25 \% abgesenkt. Lediglich in Sekunde 59 wird die Amplitude zur Erkennung der neuen Minute nicht abgesenkt. Tabelle \ref{tbl_dcf77kod} zeigt die Bedeutung der gesendeten Bits im DCF77-Signal\footnote{\cite{dcf77}, Seite 351ff.}.
%
\renewcommand{\arraystretch}{1}
\begin{longtable}{|l|l|}
\hline Bit & Bedeutung\\\hline\hline\endhead
\hline\endfoot\endlastfoot
%
0 & \\\hline
1 & \\\hline
2 & \\\hline
3 & \\\hline
4 & \\\hline
5 & \\\hline
6 & \\\hline
7 & \\\hline
8 & \\\hline
9 & \\\hline
10 & \\\hline
11 & \\\hline
12 & \\\hline
13 & \\\hline
14 & \\\hline
15 & \\\hline
16 & \\\hline
17 & \\\hline
18 & \\\hline
19 & \\\hline
20 & \\\hline
21 & \\\hline
22 & \\\hline
23 & \\\hline
24 & \\\hline
25 & \\\hline
26 & \\\hline
27 & \\\hline
28 & \\\hline
29 & \\\hline
30 & \\\hline
31 & \\\hline
32 & \\\hline
33 & \\\hline
34 & \\\hline
35 & \\\hline
36 & \\\hline
37 & \\\hline
38 & \\\hline
39 & \\\hline
40 & \\\hline
41 & \\\hline
42 & \\\hline
43 & \\\hline
44 & \\\hline
45 & \\\hline
46 & \\\hline
47 & \\\hline
48 & \\\hline
49 & \\\hline
50 & \\\hline
51 & \\\hline
52 & \\\hline
53 & \\\hline
54 & \\\hline
55 & \\\hline
56 & \\\hline
57 & \\\hline
58 & \\\hline
59 & \\\hline
\caption{Erläuterung der Bits im DCF77-Zeitsignal}\label{tbl_dcf77kod}
\end{longtable}
%
\subsection{LED Matrix}
Unter LED Matrix versteht man die Ansteurung von LEDs in Zeilen und Spalten. Dabei werden alle Annoden/Kathoden zu Spalten und alle Kathoden/Anoden zu Zeilen
verbunden. 

\subsection{Pulsweitenmodulation}\label{sec_pulsweitenmodulation}
TODO
