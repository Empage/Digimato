\section{Anforderungen}
% - Anfordungen an die Uhr
%  - elementar
%   - Zeitempfang
%   - Anzeigen der Zeit
%  - erweitert
%   - Datumsanzeige
%   - Wecker
%   - Helligkeitsanpassung
%   - geschaltet steckdose
%   - infrarotempfang
%   - temperatursensor
%   - Updatefaehigkeit
%   - Uhrzeit einstellen
In der Konzeptionsphase der Digitaluhr wurden die in den folgenden Subkapiteln beschriebenen Anforderungen definiert.
Da das Projekt ein rein privat initiiertes ist, sind alle Anforderungen selbst definiert. Damit sich das Projekt jedoch
lohnt, wurden durchaus anspruchsvolle Anforderungen mit speziellem Fokus auf Erweiterbarkeit an die Uhr gestellt, weil die
Erweiterbarkeit und Flexibilität die größten Vorteile einer Eigenkonstruktion darstellen.
%
\subsection{Modularität}
Der modulare Aufbau der Digitaluhr ist eine der wichtigsten Anforderungen. Dabei bezieht sie sich sowohl auf die Hard-,
als auch auf die Software.

Es soll darauf geachtetet werden, dass möglichst alle externen Komponenten durch Steckverbindungen mit der Hauptplatine verbunden werden. 
Dies soll dafür sorgen, dass Komponenten leicht ausgetauscht werden können, sei es aus Gründen eines Defekts oder weil sich eine 
gleiche Komponente eines anderen Herstellers als besser erweist. Zusätzlich sorgt es dafür, dass die Hauptplatine unabhängig
von den anderen Komponenten entnommen werden kann.

Auf der Softwareseite soll darauf geachtet werden, komponentenspezifischen Code in extra Funktionen auszulagern, sowie
Konfigurationseinstellungen in sinnvoll benannten Konstanten festzuhalten. Damit kann in der Zukunft Funktionalität leicht und
übersichtlich geändert oder hinzugefügt werden.
%
\subsection{Zeitempfang und Anzeige der Zeit}
Selbstverständlich gehört auch die Anzeige der Zeit zu den wichtigsten Anforderungen an eine Uhr. Die Zeit soll über eine
LED-Matrix angezeigt werden. Außerdem soll die Zeit automatisch empfangen werden können. Dafür bietet sich der 
Zeitzeichensender DCF77 an, der die deutsche Zeit und das Datum via Funk verbreitet. Es soll zusätzlich die Möglichkeit
bestehen, die Zeit manuell einzustellen, um die Uhr einerseits auch in einer anderen Zeitzone und
andererseits trotz gestörtem Empfang betreiben zu können.

Darüber hinaus soll die Möglichkeit bestehen, das Datum anzeigen (und einstellen) lassen zu können.
%
\subsection{Automatische Helligkeitsanpassung}
Die Uhr soll die Umgebungshelligkeit detektieren und die Helligkeit ihrer Anzeige dynamisch anpassen können. Sie kann sich so
durch hohe Helligkeit am Tag und Dimmen in der Nacht der Umgebung gut anpassen und sorgt damit für eine optimale Lesbarkeit.
%
\subsection{Temperatursensor}
Die Digitaluhr soll in der Lage sein, die Temperatur des Raumes messen und anzeigen zu können.
%
\subsection{Updatefähigkeit}
Diese Anforderung bezieht sich speziell auf die Software der Digitaluhr. Da eine einfache Erweiterbarkeit sowie Fehlerkorrektur
gewünscht ist, soll sich die Software leicht updaten lassen. Dazu soll das Programmierinterface ohne die fertige Uhr
aufschrauben zu müssen, verfügbar sein.
%