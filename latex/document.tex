\documentclass[%
	pdftex,
	oneside,		% Einseitiger Druck.
	12pt,			% Schriftgroesse
	parskip=half,	% Halbe Zeile Abstand zwischen Absätzen.
	headsepline,	% Linie nach Kopfzeile.
	%footsepline,	% Linie vor Fusszeile.
	abstracton,	    % Abstract Überschriften
	ngerman,		% Translator
]{scrartcl}

% Zeichencodierung
\usepackage[utf8]{inputenc}
\usepackage[T1]{fontenc}
 
\newcommand{\pdftitel}{Konzeption und Entwicklung einer digitalen Funk-, LED-Uhr}
\newcommand{\autor}{Tobias Schöneberger, Matthis Hauschild}
\newcommand{\arbeit}{Studienarbeit}
%falls pdftitel = titel der Arbeit
\newcommand{\titel}{\pdftitel}
%bei unterschiedlichen Titeln
%\newcommand{\titel}{In der Regel haben wir einen zweizeiligen
% Bachelorthesistitel}
\newcommand{\kurs}{TIT10AID}
\newcommand{\datumAbgabe}{Juni 2013}
\newcommand{\abgabeort}{Stuttgart}
\newcommand{\studiengang}{Angewandte Informatik}
\newcommand{\dhbw}{Stuttgart}
\newcommand{\betreuer}{Prof. Dr. Karl Friedrich Gebhardt}
\newcommand{\gutachter}{Prof. Dr. Karl Friedrich Gebhardt}
\newcommand{\zeitraum}{12 Wochen}
\newcommand{\arbeitsart}{\arbeit}


%Seitengroesse
\usepackage{fullpage} 

%Zeilenumbruch und mehr
\usepackage[activate]{microtype}


% Zeilenabstand
\usepackage[onehalfspacing]{setspace}

% Index-Erstellung
%\usepackage{makeidx}

% Lokalisierung (neue deutsche Rechtschreibung)
\usepackage[ngerman]{babel}  
 
% Anführungszeichen 
\usepackage[babel,german=quotes]{csquotes}
%\usepackage[style=swiss]{csquotes}


% Spezielle Tabellenform fuer Deckblatt
\usepackage{longtable}
\setlength{\tabcolsep}{10pt} %Abstand zwischen Spalten
\renewcommand{\arraystretch}{1.5} %Zeilenabstand

% Grafiken
\usepackage{graphicx}

%Farben
\usepackage{color}

%\usepackage{fancyhdr}

\usepackage{bibgerm}
\usepackage{caption}
\usepackage{longtable}
% für textumflossene Grafiken
\usepackage{wrapfig} 

\pagestyle{fancy} 

% Fussnoten
\usepackage[perpage, hang, multiple, stable]{footmisc}

% Verschiedene Schriftarten
%\usepackage{goudysans}
\usepackage{lmodern}
%\usepackage{libertine}
%\usepackage{palatino} 

% Hurenkinder und Schusterjungen verhindern
% http://projekte.dante.de/DanteFAQ/Silbentrennung
\clubpenalty=10000
\widowpenalty=10000
\displaywidowpenalty=10000

% Quellcode
\usepackage{listings}
\lstloadlanguages{Java}
\lstset{%
	language=PHP,		 	 % Sprache des Quellcodes
	%numbers=left,           % Zelennummern links
	stepnumber=1,            % Jede Zeile nummerieren.
	numbersep=5pt,           % 5pt Abstand zum Quellcode
	numberstyle=\tiny,       % Zeichengrösse 'tiny' für die Nummern.
	breaklines=true,         % Zeilen umbrechen wenn notwendig.
	breakautoindent=true,    % Nach dem Zeilenumbruch Zeile einrücken.
	postbreak=\space,        % Bei Leerzeichen umbrechen.
	tabsize=2,               % Tabulatorgrösse 2
	basicstyle=\ttfamily\footnotesize, % Nichtproportionale Schrift, klein für den Quellcode
	showspaces=false,        % Leerzeichen nicht anzeigen.
	showstringspaces=false,  % Leerzeichen auch in Strings ('') nicht anzeigen.
	extendedchars=true,      % Alle Zeichen vom Latin1 Zeichensatz anzeigen.
	captionpos=b,            % sets the caption-position to bottom
	backgroundcolor=\color{ListingBackground} % Hintergrundfarbe des Quellcodes setzen.
}

\newcommand{\degree}{\ensuremath{^\circ}}

% Titel, Autor und Datum
\title{\titel}
\author{\autor}
\date{\datum}

% PDF Einstellungen
\usepackage[%
	pdftitle={\pdftitel},
	pdfauthor={\autor},
	pdfsubject={\arbeit},
	pdfcreator={pdflatex, LaTeX with KOMA-Script},
	pdfpagemode=UseOutlines, % Beim Oeffnen Inhaltsverzeichnis anzeigen
	pdfdisplaydoctitle=true, % Dokumenttitel statt Dateiname anzeigen.
	pdflang=de % Sprache des Dokuments.
]{hyperref} 



\begin{document}
	% Deckblatt
	\begin{spacing}{1}
		\begin{titlepage}
	\begin{longtable}{p{.4\textwidth} p{.4\textwidth}}
	  {\includegraphics[height=2.6cm]{images/logo.png}} & 
	  {\includegraphics[height=2.6cm]{images/dhbw.png}}
	\end{longtable}
	\enlargethispage{25mm}
	\begin{center}
	  \vspace*{18mm}	{\LARGE\bf \titel }\\
	  \vspace*{18mm}	{\large\bf \arbeit}\\
	  \vspace*{18mm}	\studiengang\\
	  \vspace*{3mm} 	an der Dualen Hochschule Baden-Württemberg \dhbw\\
	  \vspace*{18mm}	von\\
	  \vspace*{3mm} 	{\large\bf \autor}\\
	  \vspace*{18mm}	\datumAbgabe\\
	\end{center}
	\vfill
	\begin{spacing}{1.2}
	\begin{tabbing}
		mmmmmmmmmmmmmmmmmmmmmmmmmm     \= \kill
		\textbf{Abgabe}  \>  \zeitraum\\
		\textbf{Kurs}  				   \>  \kurs\\
		\textbf{Betreuer}              \>  \betreuer\\
		\textbf{Gutachter}             \>  \gutachter
	\end{tabbing}
	\end{spacing}
\end{titlepage}

	\end{spacing}
	\newpage

	% Sperrvermerk
	\renewcommand{\thepage}{\Roman{page}} 
	\setcounter{page}{1}

	% Erklärung
	\thispagestyle{empty}

\section*{Erklärung}
% http://www.se.dhbw-mannheim.de/fileadmin/ms/wi/dl_swm/dhbw-ma-wi-organisation-bewertung-bachelorarbeit-v2-00.pdf
\vspace*{2em}

Wir erklären hiermit ehrenwörtlich: \\
\begin{enumerate}
\item dass wir unsere {\arbeitsart} mit dem Thema
{\itshape \titel } ohne fremde Hilfe angefertigt haben;
\item dass wir die Übernahme wörtlicher Zitate aus der Literatur sowie die
Verwendung der Gedanken anderer Autoren an den entsprechenden Stellen innerhalb
der Arbeit gekennzeichnet haben;
\item dass wir unsere {\arbeitsart} bei keiner anderen Prüfung vorgelegt habe;
\item dass die eingereichte elektronische Fassung exakt mit der eingereichten schriftlichen Fassung
übereinstimmt.
\end{enumerate}

Wir sind uns bewusst, dass eine falsche Erklärung rechtliche Folgen haben wird.

\vspace{3em}

\abgabeort, \datumAbgabe
\vspace{6em}

\begin{tabular}[h]{p{5cm} p{2cm} p{5cm}}
  \hline
	Matthis Hauschild & &Tobias Schöneberger \\
\end{tabular} 

	\newpage
	
	\tableofcontents
	\newpage 
	
	\renewcommand{\thepage}{\arabic{page}}
	\setcounter{page}{1}
	 
	\section{Einleitung}
\subsection{Motivation}
TODO
- vielseitig, weil:
- funkübertragung
- signalauswertung
- matrixansteuerung
- 1wirebus
- multiplexen
- mikrocontroller programmierung in c
- effizientes programmierung durch begrenzte ressourcen
- arbeit im team
- handwerkliche taetigkeit

\subsection{Umfang}
Dieser Bericht soll die Designs-, Konzeptions- sowie Entwicklungsphase der Digitaluhr dokumentieren. Dabei werden zunächst die technischen Grundlagen beschrieben, anschließend wird auf die einzelnen Komponenten eingegangen und darauffolgend das Zusammenspielen der Komponenten betrachtet. An relevanten Stellen wurde die entsprechenden Codeausschnitte angefügt. Es wurde darauf verzichtet, den kompletten Sourcecode in der Arbeit zu erläutern, weil er sich --- gut kommentiert --- im Anhang befindet. Abschließend wird das Ergebnis des Projekts im Kapitel Résumé diskutiert und kritisch betrachtet. 
	\section{Anforderungen}
% - Anfordungen an die Uhr
%  - elementar
%   - Zeitempfang
%   - Anzeigen der Zeit
%  - erweitert
%   - Datumsanzeige
%   - Wecker
%   - Helligkeitsanpassung
%   - geschaltet steckdose
%   - infrarotempfang
%   - temperatursensor
%   - Updatefaehigkeit
%   - Uhrzeit einstellen
In der Konzeptionsphase der Digitaluhr wurden die in den folgenden Subkapiteln beschriebenen Anforderungen definiert.
Da das Projekt ein rein privat initiiertes ist, sind alle Anforderungen selbst definiert. Damit sich das Projekt jedoch
lohnt, wurden durchaus anspruchsvolle Anforderungen mit speziellem Fokus auf Erweiterbarkeit an die Uhr gestellt, weil die
Erweiterbarkeit und Flexibilität die größten Vorteile einer Eigenkonstruktion darstellen.
%
\subsection{Modularität}
Der modulare Aufbau der Digitaluhr ist eine der wichtigsten Anforderungen. Dabei bezieht sie sich sowohl auf die Hard-,
als auch auf die Software.

Es soll darauf geachtetet werden, das möglichst alle externen Komponenten durch Stecker mit der Hauptplatine verbunden werden. 
Dies soll dafür sorgen, dass Komponenten leicht ausgetauscht werden können, sei es aus Gründen eines Defekts oder weil sich eine 
gleiche Komponente eines anderen Herstellers als besser erweist. Zusätzlich sorgt es dafür, dass die Hauptplatine unabhängig
von den anderne Komponenten entnommen werden kann.

Auf der Softwareseite soll darauf geachtet werden, komponentenspezifischen Code in extra Funktionen auszulagern, sowie
Konfigurationseinstellungen in sinnvoll benamsten Konstanten festzuhalten. Damit kann in der Zukunft Funktionalität leicht und
übersichtlich geändert oder hinzugefügt werden.
%
\subsection{Zeitempfang und Anzeige der Zeit}
Selbstverständlich gehört auch die Anzeige der Zeit zu den wichtigsten Anforderungen an eine Uhr. Die Zeit soll über eine
LED-Matrix angezeigt werden. Außerdem soll die Zeit automatisch empfangen werden können. Dafür bietet sich der 
Zeitzeichensender DCF77 an, der die deutsche Zeit und das Datum via Funk verbreitet. Es soll zusätzlich die Möglichkeit
bestehen, die Zeit manuell einzustellen, um die Uhr einerseits auch in einer anderen Zeitzone und
andererseits trotz gestörtem Empfang betreiben zu können.

Darüber hinaus soll die Möglichkeit bestehen, das Datum anzeigen (und einstellen) lassen zu können.
%
\subsection{Automatische Helligkeitsanpassung}
Die Uhr soll die Umgebungshelligkeit detektieren und die Helligkeit ihre LEDs dynamisch anpassen können. Sie kann sich so
durch hohe Helligkeit am Tag und Dimmen in der Nacht der Umgebung gut anpassen und sorgt damit für eine optimale Lesbarkeit.
%
\subsection{Temperatursensor}
Die Digitaluhr soll in der Lage sein, die Temperatur des Raumes messen und anzeigen zu können.
%
\subsection{Updatefähigkeit}
Diese Anforderung bezieht sich speziell auf die Software der Digitaluhr. Da eine einfache Erweiterbarkeit sowie Fehlerkorrektur
gewünscht ist, soll sich die Software leicht updaten lassen. Dazu soll das Programmierinterface ohne die fertige Uhr
aufschrauben zu müssen, verfügbar sein.
%
	\section{Technische Grundlagen}
\subsection{DCF77}\label{sec_dcfgrund}
DCF77 ist ein Zeitzeichensender in Mainhausen bei Frankfurt, der seit dem ersten Januar 1959 die Uhrzeit auf der Langwellenfrequenz von 77,5 kHz sendet. Bei Sendeanlagen, die über Ländergrenzen hinaus senden, muss das Rufzeichen in der internationalen Frequenzliste eingetragen sein und das Kennzeichen des jeweiligen Landes enthalten. Deshalb wurde DCF77 gewählt, wobei das D für Deutschland steht. Der Buchstabe C war früher ein Kennzeichen für Langwelle und das F steht für Frankfurt. Da die Sendeanlage in Mainhausen mehrere Sender hat, wurde noch die Zahl 77 als Andeutung auf die Trägerfrequenz (77,5 kHz) gewählt\footnote{\cite{dcf77}, Seite 349f.}.

Das Zeitsignal wird jede Minute wiederholt und kodiert Zeit- sowie Datumsinformationen. Dabei wird jede Sekunde ein Bit übertragen. Dies geschieht durch Amplitudenmodulation. Jede Sekunde wird die Amplitude der Trägerwelle für 100 ms (logisch 0) oder 200 ms (logisch 1) auf etwa 25 \% abgesenkt. Lediglich in Sekunde 59 wird die Amplitude zur Erkennung der neuen Minute nicht abgesenkt. Tabelle \ref{tbl_dcf77kod} zeigt die Bedeutung der gesendeten Bits im DCF77-Signal\footnote{\cite{dcf77}, Seite 351ff.}.
\newpage
%
\renewcommand{\arraystretch}{1}
\begin{longtable}{|l|l|}
\hline Bit & Bedeutung\\\hline\hline\endhead
\hline\endfoot\endlastfoot
%
0 & Minutenanfang\\\hline
1 - 14 & verschlüsselte Wetterinformationen\\\hline
15 & Rufbit: Ankündigung, wenn von Reserveantenne gesendet wird\\\hline
16 & A1: Wechsel von/zur Sommerzeit bei nächstem Stundenwechsel\\\hline
17 - 18 & Zeitzonenbits: 01: MEZ, 10: MESZ\\\hline
19 & A2: Schaltsekunde bei nächstem Stundenwechsel\\\hline
20 & S: Startbit (immer logisch 1)\\\hline
21 & Minutenbit, Wertigkeit 1\\\hline
22 & Minutenbit, Wertigkeit 2\\\hline
23 & Minutenbit, Wertigkeit 4\\\hline
24 & Minutenbit, Wertigkeit 8\\\hline
25 & Minutenbit, Wertigkeit 10\\\hline
26 & Minutenbit, Wertigkeit 20\\\hline
27 & Minutenbit, Wertigkeit 40\\\hline
28 & Prüfbit für Minuten, Even Parity\\\hline
29 & Stundenbit, Wertigkeit 1\\\hline
30 & Stundenbit, Wertigkeit 2\\\hline
31 & Stundenbit, Wertigkeit 4\\\hline
32 & Stundenbit, Wertigkeit 8\\\hline
33 & Stundenbit, Wertigkeit 10\\\hline
34 & Stundenbit, Wertigkeit 20\\\hline
35 & Prüfbit für Stunden, Even Parity\\\hline
36 & Kalendertag, Wertigkeit 1\\\hline
37 & Kalendertag, Wertigkeit 2\\\hline
38 & Kalendertag, Wertigkeit 4\\\hline
39 & Kalendertag, Wertigkeit 8\\\hline
40 & Kalendertag, Wertigkeit 10\\\hline
41 & Kalendertag, Wertigkeit 20\\\hline
42 & Wochentag, Wertigkeit 1\\\hline
43 & Wochentag, Wertigkeit 2\\\hline
44 & Wochentag, Wertigkeit 4\\\hline
45 & Kalendermonat, Wertigkeit 1\\\hline
46 & Kalendermonat, Wertigkeit 2\\\hline
47 & Kalendermonat, Wertigkeit 4\\\hline
48 & Kalendermonat, Wertigkeit 8\\\hline
49 & Kalendermonat, Wertigkeit 10\\\hline
50 & Kalenderjahr, Wertigkeit 1\\\hline
51 & Kalenderjahr, Wertigkeit 2\\\hline
52 & Kalenderjahr, Wertigkeit 4\\\hline
53 & Kalenderjahr, Wertigkeit 8\\\hline
54 & Kalenderjahr, Wertigkeit 10\\\hline
55 & Kalenderjahr, Wertigkeit 20\\\hline
56 & Kalenderjahr, Wertigkeit 40\\\hline
57 & Kalenderjahr, Wertigkeit 80\\\hline
58 & Prüfbit für Kalenderjahr, Even Parity\\\hline
59 & Markierung der neuen Minute, keine Amplitudenabsenkung\\\hline
\caption{Erläuterung der Bits im DCF77-Zeitsignal\footnote{\cite{dcf77}, Seite 352f.}}\label{tbl_dcf77kod}
\end{longtable}
%
\subsection{LED Matrix}
Unter LED Matrix versteht man die Ansteurung von LEDs in Zeilen und Spalten. Dabei werden alle Anoden zu Spalten und alle Kathoden zu Zeilen
verbunden.\footnote{Es kann natürlich auch die Kathode für Spalten und die
Anode für Zeilen verwendet werden.} Im Vergleich zu einer Einzelansteuerung
bietet die LED Matrix den großen Vorteil, dass bei einer Matrix mit $N$ Spalten
und $M$ Zeilen nur $N+M$ statt $N*M$ Leitungen verwendet werden.

\begin{wrapfigure}{r}{0.45\textwidth}
  \vspace{-25pt}
  \begin{center}
    \includegraphics[width=0.42\textwidth]{skizzen/led_matrix_5x7.png}
  \end{center}
  \vspace{-20pt}
  \captionof{figure}{5x7 LED Matrix}
\end{wrapfigure}

Die LED Matrix
wird dann zeilenweise oder spaltenweise im Multiplexbetrieb angesteuert. Das bedeutet, dass nacheinander eine der Spalten mit GND versorgt wird und die anderen Spalten unbeschalten sind (keine Verbindung zu GND). Nun können in dieser Spalte durch Anlegen von Spannung an den entsprechenden Zeilen LEDs angeschaltet werden. 
Dieser Vorgang wird für alle Spalten durchgeführt, das bedeutet es leuchten zu
einem bestimmten Zeitpunkt immer nur die LEDs einer Spalte. Durch das schnelle
Umschalten zwischen den Spalten und die Trägheit des menschlichen Auges entsteht
die Illusion, dass auf der kompletten LED Matrix die LEDs aktiviert sind.
Als Nachteil aus dieser Beschaltung ergibt sich die verringerte Helligkeit, da
die LEDs bei $N$ Spalten nur noch $\frac{1}{N}$ der Zeit leuchten.

Der Helligkeitsverlust kann durch höheren Stromfluss zum Teil kompensiert
werden. Das bedeutet bei $N$ Spalten werden die LEDs mit einem Pulsstrom von
$N*Nennstrom$ betrieben. Durch die Dunkelphasen kann das aktive Substrat
zwischen den Pulsen ausreichend abkühlen. Generell kann dies bis zum ca.
zehnfachen Nennstrom (200 mA bei einer gewöhnlichen 20 mA LED) durchgeführt
werden.\footnote{vgl. \cite{ledMatrix}}

\subsection{Pulsweitenmodulation}\label{sec_pulsweitenmodulation}
Pulsweitenmodulation bezeichnet eine Modulationstechnik in der die Weite des
Pulses bei einer gleichbleibenden Periode verändert wird (siehe \ref{pwm_schma}).
Diese Modulationstechnik erlaubt es, die Leistung von Geräten zu regulieren.
Der durchschnittliche Stromfluss wird durch das Verhältniss
$\frac{Pulsweite}{Periode}$ definiert. Gilt $Pulsweite=Periode$ so erhält das
Gerät 100\% der Leistung, gilt $\frac{Pulsweite}{Periode}=\frac{1}{2}$ so wird das Gerät mit halber Leistung
versorgt.
Die Periode wird in der Regel sehr klein gewählt, zum Beispiel $\frac{1}{100}s$
bei LEDs, da das menschliche Auge 100 Hz blinken als konstantes Leuchten
wahrnimmt.
\begin{figure}[h]
  \begin{center}
    \includegraphics[width=0.8\textwidth]{skizzen/pwm.png}
  \end{center}
  \caption{Schema der Pulsweitenmodulation}
  \label{pwm_schma}
\end{figure}



	\section{Betrachtung der Komponenten}
\subsection{Mikrocontroller}
TODO

\subsection{DCF77 Empfangsmodul}
TODO

\subsection{LED Matrix}
TODO

\subsection{Helligkeitssensor}
TODO

\subsection{Temperatursensor}
TODO

\subsection{Infrarotsensor}
TODO
	\section{Betrachtung des Gesamtsystems}
TODO
% Gehäuse
% Stromverbrauch
	\section{Résumé}
\subsection{Evaluation}
TODO

\subsection{Weiterentwicklungsmöglichkeiten}
TODO

\subsection{Fazit}
TODO
% Kosten

% Einleitung
% -Motivation
% -Umfang
% Anforderungen
% Technische Grundlagen
% - DCF77
% - LED Matrix
% - PWM
% Betrachtung der Komponenten
% - Mikrocontroller
% - DCF77 Empfangsmodul
% - LED Matrix
% - Helligkeitssensor
% - Temperatursensor
% - Infrarotempfänger
% Betrachtung des Gesamtsystems
% - Gehäuse
% Zusammenfassung
% -Evaluation
% -Weiterentwicklungsmoeglichkeiten
% -Fazit
% - Kosten


% NANANNANAN Matt\cite{cite1} Tobi\cite{cite2}
% 
% Uberblick gesamtsystem
% - Anfordungen an die Uhr
%  - elementar
%   - Zeitempfang
%   - Anzeigen der Zeit
%  - erweitert
%   - Datumsanzeige
%   - Wecker
%   - Helligkeitsanpassung
%   - geschaltet steckdose
%   - infrarotempfang
%   - temperatursensor
%   - Updatefaehigkeit
%   - Uhrzeit einstellen
%   
% -Betrachtung der einzelenen Elemente
%  - minimal
%  - erweitert
% -Display (LED Matrix)
%  - Prinzip
%  - Moeglichkeiten zur umsetzung
%  - vor und nachteile der Moeglichen
% - Zeitempfang
%  - DCF77 vs. GPS vs. England sender
%    GPS genauer aber teuerer, indoorempfang gut?
%  - DCF77 im Detail
%  - vergleich von verschieden empfangsmodulen
%  - vergleich von verschieden methoden zur auswertung
% - Mikrocontroller
%  - Warum mikrocontroller? und nicht fertiger baustein?
%  - ATmega vs. z.B. Cortex-m3 vs 
%  - Pro:
%   - einfache Programmierung
%   - guenstig
%   - einzeln verfuegbar
%   - DIP
%   - geringer stromverbrauch
%   - viele IO Pins
%   - viel verfuegbarer code
% 
%  
% - Zusatzmodule
%  - Temperatur
%   1 wire bus sensor um pins zu sparen und adc muss dann nicht umgeschaltet werden
%  - Helligkeit und dimmen
%   Lichtabhaeniger Widerstand an A/D Wandler und LED PWM
%  - Wecker
%   Lautsprecher + Weckzeit im EEPROM
%  - Steckdosenmodul
%   Relais mit verstaerkendem transistor/fet
%  - Infrarotmodul
%   Empfang und dekodierung von Infrarotsignalen von Fernbedienungen zur einstellung der Weckzeit
%  -
\newpage
\bibliography{Literatur/stud}
\bibliographystyle{geralpha}
\end{document}