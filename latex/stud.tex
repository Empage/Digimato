\documentclass[16pt,a4paper]{article}
\usepackage{ngerman} % neue Rechtschreibung + Sprachumstellung
\usepackage[utf8]{inputenc} % ASCI-8bit-Ei
\usepackage{color}
\usepackage[T1]{fontenc} % T1 encoding
\usepackage{graphicx}
\usepackage{fancyhdr}
\usepackage{bibgerm}

\pagestyle{fancy}

\title{Konzeption und entwicklung einer digitalen Funk-, LED-Uhr}
\author{Tobias Sch"oenberger, Matthis Hauschild}
\date{\today}
\begin{document}
\maketitle

NANANNANAN Matt\cite{cite1} Tobi\cite{cite2}

Uberblick gesamtsystem
- Anfordungen an die Uhr
 - elementar
  - Zeitempfang
  - Anzeigen der Zeit
 - erweitert
  - Datumsanzeige
  - Wecker
  - Helligkeitsanpassung
  - geschaltet steckdose
  - infrarotempfang
  - temperatursensor
  - Updatefaehigkeit
  - Uhrzeit einstellen
  
-Betrachtung der einzelenen Elemente
 - minimal
 - erweitert
-Display (LED Matrix)
 - Prinzip
 - Moeglichkeiten zur umsetzung
 - vor und nachteile der Moeglichen
- Zeitempfang
 - DCF77 vs. GPS vs. England sender
   GPS genauer aber teuerer, indoorempfang gut?
 - DCF77 im Detail
 - vergleich von verschieden empfangsmodulen
 - vergleich von verschieden methoden zur auswertung
- Mikrocontroller
 - Warum mikrocontroller? und nicht fertiger baustein?
 - ATmega vs. z.B. Cortex-m3 vs 
 - Pro:
  - einfache Programmierung
  - guenstig
  - einzeln verfuegbar
  - DIP
  - geringer stromverbrauch
  - viele IO Pins
  - viel verfuegbarer code

 
- Zusatzmodule
 - Temperatur
  1 wire bus sensor um pins zu sparen und adc muss dann nicht umgeschaltet werden
 - Helligkeit und dimmen
  Lichtabhaeniger Widerstand an A/D Wandler und LED PWM
 - Wecker
  Lautsprecher + Weckzeit im EEPROM
 - Steckdosenmodul
  Relais mit verstaerkendem transistor/fet
 - Infrarotmodul
  Empfang und dekodierung von Infrarotsignalen von Fernbedienungen zur einstellung der Weckzeit
 -

\bibliography{Literatur/stud}
\bibliographystyle{geralpha}
\end{document}