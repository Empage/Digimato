\section{Résumé}
\subsection{Evaluation}
TODO

\subsection{Weiterentwicklungsmöglichkeiten}
Die Grundfunktionalität der Uhr besteht nun und auch einige erweiterte Funktionen, wie beispielsweise Temperatur- und Helligkeitsmessung sind implementiert. Damit stellt das Projekt eigentlich einen abgeschlossenen Zustand dar. Dennoch gibt es natürlich Erweiterungsmöglichkeiten. Es ist vorstellbar, eine zukünftige Version mit Infrarotempfang auszustatten. So könnte man nicht nur die Zeit eingeben, die Helligkeit kontrollieren oder einen Wecker vom Bett aus bedienen, auch einfache Spiele wie das bekannte Pong wären denkbar. Eine weiteres mögliches Zusatzfeature wäre eine externe Stromversorgung einzubauen. Man könnte beispielweise bei Auslösen des Weckers die externe Stromversorgung aktivieren, sodass ein Licht oder eine Musikanlage angesteuert wird.

%Auch wenn diese Version der Uhr schon viele Funktionalität abdeckt, durch die selbstkonstruierte Bauweise sind viele Änderungen denkbar. Es wäre also durchaus der Anlass für eine zweite Version der Uhr gegeben.
%
\subsection{Fazit}
Diese Version der Uhr deckt viele Funktionalität ab. Wie aber im Kapitel Weiterentwicklungsmöglichkeiten gesehen werden kann, gibt es dennoch Raum für Erweiterungen, sodass eine Version zwei für die Uhr durchaus denkbar wäre.
% Kosten