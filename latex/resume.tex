\section{Résumé}
\subsection{Evaluation}
In diesem Teil der Arbeit soll kritisch betrachtet werden, in wie Weit die im
Teil Anforderungen gesetzten Ziele erreicht wurden.
\subsubsection{Modularität}
\vspace{-.5ex}
Die Uhr wurde Schritt für Schritt aufgebaut und hinzugekommene Funktionen
jeweils über Steckverbindungen mit der Hauptplatine verbunden. Sämtliche Module können
einzeln getauscht, ersetzt oder weiterentwickelt werden. Besonders geschickt war
dies beim Austausch des DCF-Moduls. Es konnte mit minimalen Änderungen von einem
Modul von Pollin zu dem höherwertigen Conrad Modul gewechselt werden. 
\subsubsection{Zeitempfang und Anzeige der Zeit}
\vspace{-.5ex}
Die grundlegendste Funktion, die Anzeige der Zeit, funktioniert wie geplant. Schwieriger als
gedacht hat sich der Empfang der Uhrzeit herausgestellt, da das
Empfangsmodul nicht immer korrekte Daten liefert. Durch fehlertolerante
Funktionen wird dieser Makel weitestgehend ausgeglichen und die Uhrzeit kann
korrekt empfangen werden. Die 1-Sekunden-Ticks der Uhr können, durch die Wahl
des Quarzes mit geeigneter Frequenz, mit hoher Genauigkeit eingehalten werden.
Auch die Anzeige der Uhrzeit wurde mit der LED Matrix wie gefordert umgesetzt.
Als herausragend erwies sich der geringe Stromverbrauch von durchschnittlich 3,5 Watt bei großer Helligkeit
der LEDs. Auch ermöglichten erst die LEDs eine Uhr in dieser Größe, da beispielsweise
7-Segmentanzeigen nicht in dieser Größe verfügbar sind.
%
\subsubsection{Automatische Helligkeitsanpassung}
\vspace{-.5ex}
Das Zusammenspiel von Lichtsensor, Firmware und der LED Matrix mit ihren
Helligkeitsstufen funktioniert hervorragend. Verbessert werden könnte die
Funktion nur noch durch eine Durchschnittsfunktion über die Helligkeit und
unterschiedliche Schwellwerte für das Aufhellen und Abdunkeln der
Anzeige. Dadurch würde bei einer grenzwertigen Lichtstärke
nicht immer wieder zwischen den zwei entsprechenden Helligkeitsstufen hin und
her geschalten, was momentan zu leichtem Flackern führen kann.
%
\subsubsection{Updatefähigkeit}
\vspace{-.5ex}
Die einfache Updatefähigkeit der Firmware wurde durch eine Programmierschnittstelle an der Seite der Uhr sichergestellt. So kann, ohne die Uhr aufschrauben zu müssen, jederzeit die Software korrigiert, erweitert und verbessert werden. Es könnten beispielsweise die softwareseitige Erweiterung der Helligkeitsfunktion, die im vorhergehenden Kapitel beschrieben wurde, ohne großen Aufwand implementiert werden.
%
\subsection{Weiterentwicklungsmöglichkeiten}
Die Grundfunktionalität der Uhr besteht nun und auch einige erweiterte
Funktionen, wie beispielsweise Temperatur- und Helligkeitsmessung sind
implementiert. Damit stellt das Projekt eigentlich einen abgeschlossenen Zustand
dar. Dennoch gibt es natürlich Erweiterungsmöglichkeiten. Es ist vorstellbar,
eine zukünftige Version mit Infrarotempfang auszustatten. So könnte man nicht
nur die Zeit eingeben, die Helligkeit kontrollieren oder einen Wecker vom Bett
aus bedienen, auch einfache Spiele wie das bekannte Pong wären denkbar. %Eine
% weiteres mögliches Zusatzfeature wäre eine externe Stromversorgung einzubauen.
% Man könnte beispielweise bei Auslösen des Weckers die externe Stromversorgung
% aktivieren, sodass ein Licht oder eine Musikanlage angesteuert wird.
Der ATmega verfügt noch über 6 freie I/O-Pins, die für diese und weitere Erweiterungen
zur Verfügung stehen.
%
%Auch wenn diese Version der Uhr schon viele Funktionalität abdeckt, durch die selbstkonstruierte Bauweise sind viele Änderungen denkbar. Es wäre also durchaus der Anlass für eine zweite Version der Uhr gegeben.
%
\subsection{Fazit}
Der Funktionsumfang dieser Version der Uhr geht deutlich über die Grundfunktionalitäten einer Digitaluhr hinaus. Wie aber im vorherigen Kapitel Weiterentwicklungsmöglichkeiten erläutert wird, gibt es dennoch Raum für Erweiterungen, sodass eine zweite Version der Uhr durchaus denkbar wäre.
% Kosten