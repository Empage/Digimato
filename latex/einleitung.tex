\section{Einleitung}
\subsection{Motivation}
Das Projekt einer Digitaluhr stellt sich als sehr vielseitig dar, da sowohl die
Hardware als auch die Software für das Projekt erstellt werden muss. Das
Themenspektrum umfasst den Funkempfang des Zeitsignals und dessen Auswertung,
sowie die Implementation von fehlertoleranten Algorithmen. Die Technik des
Zeitmultiplexings war zur Ansteuerung des LED Displays nötig, zudem
wurden verschiedene Sensoren verbaut, die mittels 1-Wire-Bus sowie direkt mittels
des integrierten A/D-Wandlers ausgelesen werden müssen.

Die begrenzten Ressourcen eines Mikrocontrollers stellen zudem hohe
Anforderungen an die Effizienz des Codes. Neben all diesen Punkten war auch
handwerkliches Geschick beim Aufbau des Gehäuses von Nöten. Die Arbeit im Team
erforderte enge Absprachen und gute Planung.

\subsection{Umfang der Arbeit}
Dieser Bericht soll die Design-, Konzeptions- sowie Entwicklungsphase der Digitaluhr dokumentieren. Dabei werden zunächst die technischen Grundlagen beschrieben, anschließend wird auf die einzelnen Komponenten eingegangen und darauffolgend das Zusammenspiel der Komponenten betrachtet. 
An relevanten Stellen wurden die entsprechenden Codeausschnitte angefügt. Es wurde darauf verzichtet, den kompletten Sourcecode in der Arbeit zu erläutern, weil er sich --- gut kommentiert --- im Anhang befindet. Abschließend wird das Ergebnis des Projekts im Kapitel Résumé diskutiert und kritisch betrachtet.