\section{Bilder des Gehäuses}\label{sec_gehaeusebilder}
%
\vspace{5ex}
\centerline{\includegraphics[width=\linewidth]{images/gehaeuse1.jpg}}
\captionof{figure}{Holzgitter für die LED Matrix, gefertigt aus 4mm dickem Sperrholz}\label{fig_gehaeuse1}
\vfill
\centerline{\includegraphics[width=\linewidth]{images/gehaeuse2.jpg}}
\captionof{figure}{Streichen der einzelnen Elemente des Gehäuses}\label{fig_gehaeuse2}
%
\newpage
\centerline{\includegraphics[width=\linewidth]{images/gehaeuse3.jpg}}
\captionof{figure}{Befestigung der Platine an der Rückwand}\label{fig_gehaeuse3}
\vfill
\centerline{\includegraphics[width=\linewidth]{images/gehaeuse4.jpg}}
\captionof{figure}{Befestigung der Platine an der Rückwand --- Nahaufnahme}\label{fig_gehaeuse4}
%
\newpage
\centerline{\includegraphics[width=\linewidth]{images/gehaeuse5.jpg}}
\captionof{figure}{Seitenwand mit Tastern, Sensoren sowie Programmierschnittstelle}\label{fig_gehaeuse5}
\vfill
\centerline{\includegraphics[width=\linewidth]{images/gehaeuse6.jpg}}
\captionof{figure}{Verkabelung von Platine mit LED Matrix}\label{fig_gehaeuse6}
%
\newpage
\centerline{\includegraphics[width=\linewidth]{images/gehaeuse7.jpg}}
\captionof{figure}{Knapper Abstand zwischen Platine und Drähten der LED Matrix}\label{fig_gehaeuse7}
\vfill
\centerline{\includegraphics[width=\linewidth]{images/gehaeuse8.jpg}}
\captionof{figure}{Uhr im zusammengebauten Zustand mit angeschlossenem Programmiergerät}\label{fig_gehaeuse8}
%
\newpage
\centerline{\includegraphics[width=\linewidth]{images/gehaeuse9.jpg}}
\captionof{figure}{Anzeige der Uhrzeit, aufgenommen
Nachts, bei maximal reduzierter Helligkeit}\label{fig_gehaeuse9}
\vfill

\newpage
\section{Quellcode}
\lstinputlisting[language=C,caption=main.c - Hauptfunktionen der Uhr]{sourcecode/main.c}
\lstinputlisting[language=C,caption=main.h - Prototypen der öffentlichen Funktionen aus main.c]{sourcecode/main.h}
\lstinputlisting[language=C,caption=conrad\_dcf.c - Funktionen für den Zeitempfang des DCF77 Moduls]{sourcecode/conrad_dcf.c}
\lstinputlisting[language=C,caption=conrad\_dcf.h - Prototypen der öffentlichen Funktionen aus conrad\_dcf.c]{sourcecode/conrad_dcf.h}
\lstinputlisting[language=C,caption=globals.c - Globale Variableninitialisierungen]{sourcecode/globals.c}
\lstinputlisting[language=C,caption=globals.h - Globale Variablen- und Konstantendefinition]{sourcecode/globals.h}
\lstinputlisting[language=C,caption=thermometer.c - Funktionen zum Messen der Umgebungstemperatur]{sourcecode/thermometer.c}
\lstinputlisting[language=C,caption=thermometer.h - Prototypen der öffentlichen Funktionen aus thermometer.c]{sourcecode/thermometer.h}
\lstinputlisting[language=C,caption={fontMonoSpace.h - Ein Array, welches die verwendete Schriftart enthält}]{sourcecode/fontMonoSpace.h}